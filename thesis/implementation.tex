\chapter{Implementation}
\label{sec:impl}
This chapter covers the considerations for the implementation. It is
separated according to the layered architecture. Both application
layers are represented by the planner and reasoner part, because all
other application components (e.g. for perception) can simply use the
C++ interface of the robot memory.

\section{Back-End}
\label{sec:back-end}
The back-end is realized with MongoDB. It acts as storage of the robot
memory, and executes queries. The representation of stored knowledge
utilizes the document structure of MongoDB with key-value pairs and
nested sub-documents. This allows a very flexible storage of
knowledge-pieces in the structure induced by the application and
expandable by robot memory information.

\begin{wrapfigure}{r}{0.47\textwidth}
  \vspace{-0.8cm}
\begin{lstlisting}[style=SmallJSON,
  caption={Representation of a knowledge piece in the back-end},
  label=lst:backend,
  framexleftmargin=1pt, xleftmargin=0pt,
 morekeywords={}, numbers=none]
 {
   _robot_memory_info:
   {
     persistent: true,
     decay_time: ISODate(
       "2016-05-19T 23:50:00.000Z")
   },
   type: "object info",
   name: "milk_1",
   position: {x:2.5, y:1.0, z:0.0},
   storage_place: "refrigerator"
 }
\end{lstlisting}
\vspace{-14mm}
\end{wrapfigure}
An example document stored in the back-end is shown in
\reflst{lst:backend}.  It contains knowledge about a bottle of milk as
it could be needed by a planner with additional information where it
should be stored later. Additionally the document contains meta
information needed by the robot memory (e.g. if the document should be
stored persistently or removed at restart and when it should be
dropped because it's use-time has expired). The MongoDB back-end also
manages the distribution of knowledge between multiple robots. A part
of the robot memory is only locally relevant. The other part that
should be shared has to be set up as MongoDB replica set as already
described in \refsec{sec:mongodb}. This ensures that efficient
networking code is used. The other problems of a distributed database,
such as consistency, master election, and synchronization are solved
by MongoDB using replication. The back-end also contains the
operations log (oplog) of MongoDB, a separate
%% capped
collection containing a list of changes to the database with a
timestamp. This can be accessed by the robot memory to analyze
changes for event-triggers.

\section{Robot Memory}
\label{sec:impl-memory}
The robot memory middle-layer implements most functions exceeding a
typical database.
%% between the
%% applications storing and querying knowledge and the MongoDB back-end
%% is an important part of the thesis because most of the functions
%% exceeding a typical database are realized here.
A central question is which query language should be used between
applications and the robot memory because this determines the
expressiveness and has a large impact on the performance. We choose to
use the query language of MongoDB as it is already used between the
robot memory and the back-end. This has many advantages compared to
using other query languages, such as SQL, SPARQL,
XQuery~\cite{query-languages}, and JSONiq~\cite{jsoniq}.
\todo{move into conceptual part}
%
For the application view, the query language of MongoDB is a good
choice because it is an intuitive query language, has been proven as
efficient for usual and well designed queries, and is also highly
expressive when using additional JavaScript functions or the MapReduce
paradigm~\cite{mongodb,RoboDB}. For the robot memory, it requires no
translation before application on the database and is very flexible
for extending and modifying queries because queries are structured as
documents with key-value fields and can be nested or executed in
sequence. Furthermore, MongoDB queries can easily be parsed (e.g. from
a string) by using the MongoDB C++ API. The resulting object can be
analyzed and modified for example to add key-value pairs or to check
if computables are queried.

This provides the starting point for the Storage-, Update-, and
Query-manager. When adding new documents, the
\texttt{robot\_memory\_info} sub-document can be added to store
additional meta information. When querying documents, this information
can be removed by using a filter. To detect queries for computables,
the manager can analyze the fields of the query to check if there is a
computable provided for it.  For example when a query contains the
field \texttt{type:"distance"} and there is a application that
provides computation functions for distances between two objects, the
query can be forwarded with the additional parameters, the two
objects.  When some fields are missing (e.g. only one object is given)
the application may have to compute all distances, so that the query
can afterwards be executed on the set of results (e.g. to find the
nearest one with some property). To execute the query as a usual query
with arbitrary filters, aggregation, and functions, the computation
function writes the result into a separate collection the query
can be executed on with MongoDB. To lower the computational effort of
processing the same query frequently it would be possible to add
time-bounded caching for computed knowledge. We have verified this
concept of computables using a prototype, which executes queries from
interface messages either on the MongoDB directly or in case of
computables about other interfaces, generates the knowledge from the
blackboard on demand and then executes the query.

To implement event-triggers, a registration function needs to be
provided that takes a notification function and a query to define the
event. This query can check the oplog first whether there are changed
documents with relevant key-value pairs and afterwards execute a more
complex query on the database. The registration also specifies whether
the callback function is called if the query result changes, returns
no document, or is not empty. Whether event-triggers on computables
perform well, has to be evaluated. It would be possible to allow
events only on knowledge in the database or to compute the knowledge
and check for the event in certain intervals while monitoring the
computation time.

%% The robot memory . For
%% implementation, they can be called periodically or with hook-points at
%% major steps of the robot memory such as initialization and query
%% execution. For example, a knowledge decay module could use cronjob to
%% remove documents with exceeded lifetimes in certain intervals. Further
%% hook-points would be at query-modification time (e.g. to add
%% additional meta-information) and at query-result-return time (e.g. to
%% filter or modify resulting documents).

\todo[inline]{initialization with dump}

\section{Planner/Reasoner}
\label{sec:impl-planner}
The implementation on the application layer includes the development
of the Interface Provider and the usage of the robot memory in the
planning language. As an example for the various planners and
reasoners, we focus here on CLIPS. The Interface Provider for CLIPS
can be realized as CLIPS-feature in Fawkes as it was already done for
providing CLIPS access to the blackboard, Protobuf-messaging and the
navgraph. The CLIPS \emph{robot-memory feature} will be implemented in C++
and provides the CLIPS environment with functions that call C++
functions of the robot-memory feature. For example \reflst{lst:clips-rm}
\begin{figure}
  \begin{lstlisting}[showlines,style=ReallySmallCLIPS, caption={CLIPS function to execute a query},
  label=lst:clips-rm,
  emph={skill, args, state, target, res},
  emphstyle=\bfseries\color{green!80!black},
  emph={[2]\?skill, \$\?args, wait-for-lock, \?target, use,
  WAIT-FOR-LOCK, SKILL-EXECUTION, running},
  emphstyle={[2]\bfseries\color{blue!80!black}},
  morekeywords={retract, assert, modify, skill-call, skill-to-execute,
    wait-for-lock}]
(rm-query "database.collection"
          (str-cat "{type:'order', end-time:{$gt:" ?gametime "}}"))
\end{lstlisting} %$ This is just to fix Emacs highlighting due to dollar sign in code above
\end{figure}
would be the CLIPS function, which queries all orders that have not
ended yet, by creating the query using string concatenation to fill in
the current game-time. The result would be a list of pointers to document
objects represented as instances of CLIPS templates. Similarly there
can be functions for registering events and providing computables.

The Domain Creator for CLIPS can be implemented by creating the initial
fact-base with a static list of facts and additional facts resulting
from a set of queries. For example, a domain creator implemented in
the CLIPS language could execute the query function in
\reflst{lst:clips-rm} and assert all orders as facts into the fact
base.
%% possibility to also generate rules form robot memory

CLIPS could use event-triggers to be notified when there is a new PDDL
plan in the robot memory that should be executed or when the world
model has changed in the robot memory and thus facts in CLIPS have to
be updated.

\todo[inline]{examples template engine for implementation of theoretical mapping function (e.g. subdocument into function term)}