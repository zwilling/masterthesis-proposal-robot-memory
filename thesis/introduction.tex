\chapter{Introduction}
\label{chap:introduction}
\todo[inline]{Page numbers before table of contents}
\todo[inline]{write intro, this is only the proposal into}
\color{gray}

As autonomous robots are about to advance into more and more complex
and unstructured domains, multiple challenges arise. One of the
challenges is allowing them to grasp their environment and make smart
decisions based on their knowledge. To solve this, robots have to
memorize observations and gained knowledge about the environment and
provide this knowledge to reasoning and planning systems. When
multiple reasoning and planning systems are used and when multiple
robots should work in collaboration, memorized knowledge has to be
shared. The aim of the proposed thesis is to design and implement a
robot memory that is capable of storing and querying knowledge,
sharing it between multiple systems and robots, and allowing hybrid
reasoning on it.

%% In the science fiction comedy-drama Robot \&
%% Frank~\cite{robot-and-frank}, a domestic service robot supports and
%% takes care of the demential, old man Frank to allow him a happy and
%% meaningful life. The robot cooks healthy food, helps in the household,
%% and treats mental issues by encouraging Frank to work on challenging
%% projects. To realize this vision of intelligent and helpful robots, a
%% couple of challenges have to be solved. The proposed thesis focuses on
%% the challenges of building a robot memory required in many
%% applications. For example domestic robots serving breakfast need a
%% robot memory to know which items belong on the table, where they are
%% stored, and how to clean up afterwards. Intelligent factory robots
%% need to know where resources are placed, what products are ordered,
%% and which actions need to be executed to reach a goal.

Basic requirements of a robot memory are representing, storing,
and querying knowledge. With knowledge, we mean the generalization of
the Data-Information-Knowledge-Wisdom (DIKW)
hierarchy~\cite{DIKW}. For example the DIKW hierarchy ranges from
point clouds (data) to cluster positions (information), derived object
positions (knowledge), and learned probability distributions
(wisdom). We want to memorize spatio-temporal knowledge, such as
positions with time-stamps, as well as symbolic knowledge (e.g. the
room a robot is in). Memorized knowledge is used by planners and
reasoners, knowledge based systems (KBS), to reason about the
environment and plan actions. KBS are often specialized and have
different strengths. For example PDDL based planners find plans
achieving a goal, CLIPS reasoners can execute plans and update
world models, and motion planners avoid collisions during grasping and
driving. For complex tasks and environments, different KBS can be
combined to utilize their specialized strengths. However this requires
common knowledge (e.g. so that a global planner can use the current
world model updated by a reasoner).
%
A general and shared robot memory, as intended in this thesis, can
centrally manage the robot's knowledge and supply or generate
specialized knowledge bases. This allows planners and reasoners to be
closer integrated and have world models consistent to each other
because they can utilize shared knowledge and state estimations only
have to be done once. Supplying or generating a specialized knowledge
base requires filtering the robot memory, when only a part is
relevant, and transforming the knowledge into the required form to be
usable in the planner or reasoner language. The robot memory
simplifies hybrid reasoning by providing spatio-temporal and symbolic
representations of knowledge. It is also required to provide a
persistent and scalable long time storage.
%% to be usable in complex environments,
%% where the robot should keep the memory during a restart.
%
It can also be used for multiple robots sharing common knowledge for
proper coordination, which is usually laborious to implement KBS
programming languages.

Two main features of the robot memory are \emph{computables} and
\emph{event-triggers}. The concept of computables allows computing
knowledge on demand instead of storing it statically in the robot
memory. Imagine a query for the distance of two objects. This can be
computed rather quickly but storing all distances between each two
dynamic objects and keeping them up do date is a lot of unnecessary
effort. Moreover computables allows deriving symbolic from
spatio-temporal knowledge. Event-triggers should notify components
when a previously defined conditions of the robot memory apply. This
can be used to keep a world model updated without polling or to wait
for complex situations. For example a reasoner executing a production
plan could register to be notified when a machine is out of order,
what makes the plan is infeasable.

Our contributions are a conceptual and architectural design of a
generic robot memory that is capable of efficiently storing arbitrary
symbolic or geometric information including spatio-temporal data. As a
central component it allows to consistently share knowledge for one
among different reasoning and planning systems and for another among
multiple robots. Basis for the thesis will be an implementation of the
robot memory based on the document-oriented database MongoDB using the
Fawkes Robot Software Framework and integrating specific adapters for
the CLIPS reasoning and PDDL planning systems. The thesis focuses on
two applications, which can profit from the robot memory and will be
used for evaluation. On the one hand, the robot memory should be used
by logistics robots of the RoboCup Logistics League (RCLL). Here it
should share the world model between reasoners running on multiple
robots collaborating in a team and provide the shared world model in
PDDL for the use in a global planner. On the other hand, the robot
memory should be used on a domestic service robot related to the
RoboCup@Home league. Here the robot memory is needed as flexible and
scalable long-time storage used by different components depending on
either spatio-temporal or symbolic representations of stored
knowledge.

In the following, the proposal gives an overview over the background
of the thesis with the RoboCup application domain and the used
software in \refsec{sec:background}. \refsec{sec:related} presents
related work and in \refsec{sec:approach} the approach of the thesis
is shown with the goals of the robot memory, it's theoretical
foundation, the architecture, and implementation. Evaluation
considerations are presented in \refsec{sec:eval}. We conclude with a
summary and the time-schedule in \refsec{sec:conclusion}.

\color{black}