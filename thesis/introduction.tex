\chapter{Introduction}
\label{chap:introduction}
Artificial intelligence and robotics are making a fascinating progress
and allow autonomous robots to advance into more complex everyday domains.
% Deep neural networks allowed the program \emph{AlphaGo} to win
% against highest ranked Go players~\cite{alphago}, natural language
% processing and reasoning helped the question answering system
% \emph{Watson} to win the quiz show \emph{Jeopardy!}~\cite{watson},
% and motion control allows robots to walk offroad~\cite{bigdog}.
Applications with high significance for our society include domestic
service robots supporting elderly care, autonomous cars avoiding
accidents, and intelligent factory robots allowing flexible and
efficient production. To realize these intelligent and helpful robots,
a couple of challenges have to be tackled. One of these is enabling
robots to grasp their environment and make decisions based on their
knowledge. There, robots could benefit from memorizing gained information
about their environment and providing it to reasoning and planning
systems. For example, domestic robots serving breakfast need a memory
to know which items belong on the table and where they are
stored. Intelligent factory robots need to remember where products are
in the factory and what other robots are doing.

This thesis focuses on developing such a \emph{robot memory}. Basic
requirements are representing, storing, and querying information. The
information can contain sensor data, recognized object positions,
symbolic knowledge, and beliefs. Thus the representation has to be
flexible enough to store symbolic and spatio-temporal information. The
representation we chose is \emph{document-oriented}. Pieces of
information are bundled as \emph{documents}, which are sets of
\emph{key-value pairs} with nesting and data typing. As
back-end storage system, we use the document-oriented database
\emph{MongoDB}. This has multiple advantages we need for the robot
memory, for example long time storage and the possibility of
distribution.

Another requirement is providing the information stored in the robot
memory to \emph{knowledge-based systems (KBS)}, such as planners and
reasoners. These systems need the information to reason about the
environment, plan actions, and monitor plan execution. Because there
are multiple KBS with different strengths and purposes, we want to
combine them to create a smart robot system. For example, \emph{PDDL}
based planners can find sequences of actions required to achieve a
goal. \emph{CLIPS} reasoners can execute plans and update the world
model according to observations. Motion planners, such as
\emph{OpenRAVE}, find collision free paths for grasping or driving
actions. These systems need to collaborate and exchange
information. By storing and sharing the information with a general
robot memory, we can ensure that they work on a common and consistent
world model. The separation of storing information and working with it
also allows easier exchange of components. Furthermore, a component
focused on holding the memory can provide more benefits, such as
scalability and sophisticated querying.

Because different components using the robot memory have special
purposes and internal concepts, they require different views on the
robot memory. On the one hand, they need a filter to only get relevant
information. On the other hand, they require the information to be in
the appropriate format. For the example of the robot position, a
motion planner needs it as geometric coordinates and a symbolic planner needs
it as string describing the location (e.g. \texttt{"kitchen"}). The robot memory
provides these views and supports hybrid reasoning by transforming
between spatio-temporal and symbolic information. Views are provided via
interfaces to multiple KBS. For example, the PDDL interface can
generate files describing the state of the environment, which is
represented in the robot memory.  To transform between spatio-temporal
and symbolic information and to enrich query results, we introduce the concept of
\emph{computables}, which allow computing information on demand. The
robot memory analyzes queries to figure out if information covered by
a computable is necessary for answering. In this case, it calls the
component providing the information by a computation
function. Computables also allow interfacing perception and save
computational effort. Imagine a query for the distance of
two objects. This can be computed quickly, but storing all distances
between each two objects and keeping them updated would be
inefficient.
%
Another feature of the robot memory is called \emph{trigger} and
notifies about changes. This can be used to keep a world model updated
without polling, or to wait until certain information is
available. For example, a reasoner executing a production plan could
register to be notified when there is a new order requiring
re-planning. The robot memory is also usable in a multi robot
system and thus distributable.

Our contributions are the conceptual and architectural design of this
generic robot memory and the implementation in the Fawkes robot
software framework. In one sentence, our robot memory can be described
as document-oriented database for hybrid reasoning and knowledge
sharing between multiple, potentially distributed, KBS on mobile robots. The thesis focuses on
two applications, which benefit from the robot memory and are used for
evaluation. Firstly the robot memory is used by in the RoboCup
Logistics League to share a world model between reasoners running on
multiple logistics robots. Secondly it is used in a block stacking
scenario inspired by the RoboCup@Home league, where it is used for
hybrid reasoning and collaboration between a symbolical
planner, a reasoner executive, and a motion planner.

In the following \refchap{chap:background}, we give an overview of the
context and background of the thesis and the RoboCup application
domains. \refchap{chap:related} presents related work about similar
approaches and methods.  \refchap{chap:approach} introduces the
approach of this thesis with the goals of the robot memory, its
theoretical foundation, and the architecture. The implementation of
our robot memory and the usage in application scenarios is covered in
\refchap{chap:impl}. \refchap{chap:evaluation} presents quantitative and
qualitative evaluation results and in \refchap{chap:conclusion}, we
conclude with a summary and future work.
