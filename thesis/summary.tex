\chapter{Conclusion}
\label{chap:conclusion}
In this chapter we conclude the thesis with a summary in
\refsec{sec:summary} and an outlook to possible future work in
\refsec{sec:future-work}.

\section{Summary}
\label{sec:summary}
In this thesis, we have designed and implemented a document-oriented
robot memory for knowledge sharing and hybrid reasoning on mobile
robots. The robot memory allows robots to advance into more complex
and unstructured domains by providing a generic and capable
information storage, which is the basis for memorizing a complex
environment and thus for reason in it. It can represent complex
information and expressive queries. As collaboration platform for KBS,
it provides a common and consistent information basis. Furthermore, it
simplifies hybrid reasoning by storing spatio-temporal information and
allowing transformation into symbolic information on demand.

In the theoretical foundation, we specified how information can be
represented in documents and how these can be mapped into KBS. The
theoretical foundation also describes the concept of computables,
which allows components to provide computation on demand. By using
query matching, the robot memory checks if information covered by the
computable is queried and then invokes the function of the component
providing the computable. As discussed in the evaluation, computables
are convenient for transforming hybrid information and for interfacing
perception.
%
Another important concept is notification about changes in the robot
memory with triggers. The robot memory searches the Oplog of MongoDB
with the query defining the trigger event and invokes the callback
function if there is a matching change. Triggers are useful for
keeping internal world models of KBS updated. They also have turned
out to be convenient for message passing.

In contrast to other approaches from related work, our robot memory
supports sharing information in a multi-robot system. We built a
robust, consistent, and efficiently distributed memory by utilizing
Replication Sets of MongoDB and distinguishing between a local and a
shared database. In this context, triggers are useful for notifying
about updates by other robots.
%
Based on the layered architecture, we implemented the back-end
database with MongoDB and the core robot memory containing the
operation-, trigger-, and computable-manager. To provide the robot
memory to reasoners and planners, we implemented interface providers
for PDDL, CLIPS, and OpenRAVE. The PDDL interface can generate problem
descriptions from a template and the information contained in the
robot memory. The CLIPS interface provides CLIPS functions for robot
memory operations and can map between documents and facts by comparing
key-value pairs to fact template definitions. The OpenRAVE interface
builds the motion planner scene from object positions and types
represented in the robot memory.

We used the robot memory in two application scenarios for knowledge
sharing and hybrid reasoning. Both scenarios showed that the concepts
of the robot memory are beneficial. In the RCLL, the robot memory
distributes the world model between multiple robots, which use
triggers. Important benefits are the robust, simple, and efficient
world model synchronization and the persistent storage. In the blocks
world scenario, the robot memory is the basis for the collaboration of
perception, a PDDL planner, a CLIPS executive, and the motion planner
OpenRAVE. Here, the main benefits are the world model sharing between
planners and reasoners of different types and the hybrid reasoning
support with computables.  As presented in the quantitative
evaluation, the robot memory works efficient and only adds a small
overhead compared to raw MongoDB operations. The increase in
computation time for complex queries and computables is reasonable and
caused by the computational complexity of the queries themselfs.

An important challenges we faced in the thesis is the conceptual
tuning of robot memory concepts to achieve a balance between
expressiveness and computation time. For example, we decided that
triggers can only be evaluated on single changes of the robot memory
instead of the whole set of documents. For queries and computables, we
decided to allow all features of the MongoDB query language for high
flexibility and expressiveness and make the application responsible
for efficient query formulation.  Further challenges are the design
and implementation of on demand computation in a document-oriented
system and making the representations between CLIPS and MongoDB
compatible. This is due to CLIPS using strings and symbols, which are
both mapped to string values in documents, but still have to be
distinguishable.

Concluding, this thesis has designed and implemented a very useful,
scalable, and persistent robot memory for hybrid reasoning and
knowledge sharing. Major benefits are representation of
complex information and expressive queries with document-orientation,
hybrid reasoning with computables, and world model sharing between
knowledge based systems.

\section{Future Work}
\label{sec:future-work}

meta data for belief/certainty
Running master thesises Matthias, Björn + cite them
Connecting (common sense) reasoners
Future Work Tim
computables adding kv pairs and remove them later