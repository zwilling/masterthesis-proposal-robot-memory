\chapter{Conclusion}
\label{chap:conclusion}
\section{Summary}
\label{sec:summary}
\todo[inline]{remove proposal stuff + write}
In the proposed master thesis, we develop the conceptual and
architectural design of a generic robot memory and implement it in
Fawkes with MongoDB. The envisioned robot memory is a generic,
capable, and efficient long-time storage for knowledge, which is the
basis for memorizing a complex environment and thus for reasoning in
it. The robot memory is a knowledge sharing system, on the one hand,
for multiple knowledge based systems and, on the other hand, for
multiple robots working collaboratively. Therefore it allows a
consistent and tight integration of planners and reasoners while
avoiding multiple state estimations. The robot memory is a hybrid
reasoning tool that allows storing spatio-temporal knowledge and
transforming it on demand into symbolic knowledge.

In contrast to existing knowledge processing systems, such as KnowRob
and ORO, the concept of the robot memory decouples the storage of
knowledge and reasoning on it and is based on a document-oriented
representation. As we want to show in the evaluation of the thesis,
this allows more efficient and scalable querying and tighter
integration of multiple KBS and robots teams.

The architecture of the robot memory builds on top of a database
system. It contains components for dispatching and enhancing queries
and storage requests, for knowledge computed on demand, and for
event-triggers notifying about changes. There are interface
components, which provide the functionality of the robot memory for
KBS by giving access in the used languages and transforming query
results into the appropriate form. Applications using the robot memory
can store and query knowledge, provide computables for other
components, and use event triggers.

Challenges of the thesis include the conceptual tuning of the
document-oriented knowledge representation and querying to achieve a
balance between expressiveness and computation time. Further
challenges are the design of computables in a document-oriented system
to allow on demand computation of knowledge in an efficient and still
powerful way, and the robust knowledge sharing in a distributed
multi-robot system.

The main part of the thesis focuses on the robot memory as back-end of
a robot system and foundation for future applications, such as a
thesis in preparation about global PDDL planning in the RCLL with
CLIPS as executive. Nevertheless we will use the RCLL and RoboCup@Home
application domains to implement prototypes utilizing the robot memory
and working as proof of concept for qualitative evaluation and basis
for quantitative evaluation. For these two application domains, we
will connect the robot memory to CLIPS, PDDL, and various perception
components.

\section{Future Work}
\label{sec:future-work}
