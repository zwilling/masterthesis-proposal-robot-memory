\chapter{Conclusion}
\label{chap:conclusion}
In this chapter, we summarize the thesis in \refsec{sec:summary} and
give an outlook to possible future work in \refsec{sec:future-work}.

\section{Summary}
\label{sec:summary}
In this thesis, we have designed and implemented a document-oriented
robot memory for knowledge sharing and hybrid reasoning on mobile
robots. The robot memory allows robots to advance into more complex
everyday domains by providing a generic and capable
information storage, which is the basis for memorizing a complex
environment and thus for reasoning about it. It can represent complex
information and expressive queries. As collaboration platform for KBS,
it provides a common and consistent information basis. Furthermore, it
simplifies hybrid reasoning by storing spatio-temporal information and
allowing transformation into symbolic information on demand.

In the theoretical foundation, we specified how information can be
represented in documents and how these can be mapped into KBS. The
theoretical foundation also describes the concept of computables,
which allows components to provide computation on demand. By using
query matching, the robot memory checks if information covered by the
computable is queried and then invokes the function of the component
providing the computable. As discussed in the evaluation, computables
are convenient for transforming hybrid information and for interfacing
perception.
%
Another important concept is notification about changes in the robot
memory with triggers. The robot memory searches the Oplog of MongoDB
with the query defining the trigger event and invokes the callback
function if there is a matching change. Triggers are useful for
keeping internal world models of KBS updated. They also have turned
out to be convenient for message passing.

In contrast to other approaches from related work, our robot memory
supports sharing information in a multi-robot system. We built a
robust, consistent, and efficiently distributed memory by utilizing
Replication Sets of MongoDB and distinguishing between a local and a
shared database. In this context, triggers are useful for notifying
about updates by other robots.
%
Based on the layered architecture, we implemented the back-end
database with MongoDB and the core robot memory containing the
operation-, trigger-, and computable-manager. To provide the robot
memory to reasoners and planners, we implemented interface providers
for PDDL, CLIPS, and OpenRAVE. The PDDL interface can generate problem
descriptions from a template and the information contained in the
robot memory. The CLIPS interface provides CLIPS functions for robot
memory operations and can map between documents and facts by comparing
key-value pairs to fact template definitions. The OpenRAVE interface
builds the motion planner scene from object positions and types
represented in the robot memory.

We used the robot memory in two application scenarios for knowledge
sharing and hybrid reasoning. Both scenarios showed that the concepts
of the robot memory are beneficial. In the RCLL, the robot memory
distributes the world model between multiple robots, which use
triggers. Important benefits are the robust, simple, and efficient
world model synchronization and the persistent storage. In the blocks
world scenario, the robot memory is the basis for the collaboration of
perception, a PDDL planner, a CLIPS executive, and the motion planner
OpenRAVE. Here, the main benefits are the world model sharing between
planners and reasoners of different types and the hybrid reasoning
support with computables.  As presented in the quantitative
evaluation, the robot memory works efficiently and only adds a small
overhead compared to raw MongoDB operations. The increase in
computation time for complex queries and computables is reasonable and
caused by the computational complexity of the queries themselves.

An important challenge we faced in the thesis is the conceptual
tuning of robot memory concepts to achieve a balance between
expressiveness and computation time. For example, we decided that
triggers can only be evaluated on single changes of the robot memory
instead of the whole set of documents. For queries and computables, we
decided to allow all features of the MongoDB query language for high
flexibility and expressiveness and make the application responsible
for efficient query formulation.  Further challenges are the design
and implementation of on demand computation in a document-oriented
system and making the representations between CLIPS and MongoDB
compatible. This is due to CLIPS using strings and symbols, which are
both mapped to string values in documents, but still have to be
distinguishable.

Concluding, this thesis has designed and implemented a useful,
scalable, and persistent robot memory for hybrid reasoning and
knowledge sharing. Major benefits are representation of
complex information and expressive queries with document-orientation,
hybrid reasoning with computables, and world model sharing between
knowledge-based systems.

\section{Future Work}
\label{sec:future-work}
Because the robot memory provides the back-end information storage of
a robot system, it is meant to be the foundation for other components
using it and thus for future work. Especially planners and reasoners
benefit from collaboration possibilities and the hybrid reasoning
capabilities of the robot memory. There are two ongoing master theses,
which already use the robot memory and implement new global planning
approaches in the RCLL~\cite{bjoern-thesis,matthias-thesis}. Both utilize the robot memory as common
storage of the world model, which is updated by the CLIPS executive
and read by a central planner. Furthermore, they use the robot memory
to provide plans to the executive with triggers. Both also base their
implementation on prototypes developed in the application scenarios of
this thesis. On the one hand, the RCLL application scenario provides
the distributed memory, which can be accessed by their planners and is
kept updated by the CLIPS executive. On the other hand, the blocks
world scenario provides a prototype showing how a CLIPS executive can
fetch plans from the robot memory with triggers, represent it as task,
and execute it step by step. The first thesis by Björn Schäpers uses reactive answer set
programming with real-time constraints~\cite{bjoern-thesis} and the
second one by Matthias Löbach uses a PDDL planner and converts the resulting plan into a
simple temporal network~\cite{matthias-thesis}. The second one also
makes use of our PDDL model generation and considers using
computables for travel distance calculations.

Future work on the robot memory itself can further improve it in
various ways. The robot memory can be extended by more interfaces for
KBS, such as Golog, which can also benefit from the theoretical
foundation. Furthermore the robot memory can be improved internally by
incorporating confidence values or more knowledge representation
concepts, such as beliefs. Potentially beneficial MongoDB features
that we did not cover are geospatial indexing and queries. These allow
to query documents with location information by proximity or
intersection. A possible improvement of computables is the addition
of computed key-value pairs in already existing documents.
%\todo[inline]{mention integration of Deebul}

As part of Fawkes, the robot memory is open source and thus available
for many future projects.
