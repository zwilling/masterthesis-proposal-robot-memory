\chapter{Related Work}
\label{sec:related}
This section gives an overview of the related work of the proposed
thesis starting with the knowledge processing systems KnowRob and
OpenRobots Ontology, which provide similar functionality. Afterwards
we present the Generic Robot Database based on MongoDB and Fawkes, and
Open-EASE, a combination of KnowRob and the Generic Robot Database.

\section{KnowRob}
\label{sec:knowrob}
KnowRob is an open source knowledge processing system for
cognition-enabled robots~\cite{KnowRob,KnowRob-Representation}. It is
designed to store knowledge about the environment and set it into
relation to common sense knowledge to understand vaguely described
tasks, such as "set the table". For representation and inference of knowledge,
KnowRob uses Description Logic and approaches of the semantic web.
%% , an effort to make websites machine understandable,
Each small piece of information is represented as a \emph{Resource
  Description Framework (RDF) triple} (e.g. $rdf(robot, holding,
cup)$) that are composed of a subject, predicate, and
object. \todo[inline]{rdf already introduced in background}A set of
these RDF triples in a network can compose commonsense knowledge based
on the \emph{Web Ontology Language (OWL)}. It uses \emph{ontologies},
which can roughly be described as directed graphs setting objects or
classes of objects into relation. The graph can be represented with
multiple RDF triples, where the subject and object are nodes and the
predicate is a directed edge from the subject to the object. In these
ontologies, it is possible to represent common sense knowledge, such
as "milk is-a perishable", "refrigerator storage-place-for
perishable", and "refrigerator is-a box-container". Thus a robot asked
to bring milk could infer that the milk is stored in the refrigerator,
which needs to be opened first since it is a box-like container. To be
able to interface with perception, KnowRob uses a concept called
\emph{virtual knowledge base}. It allows computing needed information
on demand instead of storing everything providently. Thus special
queries can be forwarded to other components that compute the answer
efficiently.
%% Especially for perception based on sensor data or
%% transformation of spatial relations, this is advantageous compared
%% to continuously computing the results for all possible queries and
%% storing them.
The concept inspired the usage of computables in this thesis.  The
implementation and interface of KnowRob is based on the logic
programming language Prolog, what makes representing RDF triples and
inference with ontologies intuitive.
%% KnowRob implements it with Prolog predicates
%% called \emph{Computables} which call C++ functions of other
%% components. This indeed slows down the computation of a query but is
%% more efficient than continuously computing and storing the results or
%% implementing and computing the other components algorithms in Prolog.
To understand the robots environment, large and detailed ontologies are
necessary. KnowRob features acquiring and connecting these from
various sources, such as online common sense databases, the cloud-based
robot infrastructure RoboEarth~\cite{roboearth} and websites for
deriving object information from internet shops and action recipes
from how-to websites giving step by step instructions.
However, it has been found that this acquired
knowledge needs a lot of manual extension and verification before it
can be used~\cite{KnowRob-Web}. Encyclopedic knowledge often lacks
action information needed by the robot. %% (e.g. how to grab tools) and
%% action recipes require human text understanding
%% \todo{executive integration}
%
Furthermore large ontologies limit the performance of KnowRob because
of the Prolog implementation and general complexity. In contrast to
the robot memory intended in this thesis, KnowRob focuses on common
sense reasoning instead of an efficient and flexible on-line
back-end. It does not support sharing knowledge between multiple
robots and has a too strong focus on data only usable for reasoning
components. Although it can also represent spatio-temporal knowledge,
we have performance and scalability concerns with Prolog handling
larger data-sets (e.g. for storing locations of found object over long
time to learn the distribution where they can be found).

\section{OpenRobots Ontology (ORO)}
\label{sec:oro}
The OpenRobots Ontology (ORO) is an open source knowledge processing
framework for robotics~\cite{Oro}. It is designed to enable human-robot
interaction by featuring commonsense ontologies similar to KnowRob and
modeling the human point of view. It also uses RDF triples and OWL ontologies.
The triple storage is implemented in Jena, a Java toolkit for RDF, and the
inference with Pellet, a Java OWL reasoner. Besides knowledge storing,
querying and reasoning, ORO features a modular architecture between
the back-end storage and front end socket server for querying. These
modules add features such as events that notify external components
about changes.
%%  (e.g. if instances of certain classes are added or modified).
%% \todo{executive integration}
Another module adds representations of alternative perspectives that
should allow understanding the point of view of other agents.
For example when a human asks a robot to bring the
cup, there are two cups on a table and the robot should infer which
cup is meant by knowing that the other cup is occluded from the human's
point of view. Furthermore ORO features categorization to find
differences between objects (e.g. to ask if the blue cup or the red
cup was meant in a vague task description) and memory profiles for
distinguishing long-term knowledge and short-term knowledge that is
removed when the lifetime of a fact expires. Although these are useful
features and we want to realize the concepts of events and memory
profiles in this thesis, we do not use ORO as a basis. Due to its
focus on human-robot interaction and commonsense reasoning, it is not
suitable for representing large amounts of data for non-reasoning
components because all data would have to be stored in RDF triples and
processed by the OWL reasoner. Furthermore, it does neither support
synchronizing a part of the knowledge with other robots nor knowledge
computable on demand.

\section{Generic Robot Database with MongoDB}
\label{sec:mongo-logging}
There already is a generic robot database developed with MongoDB as
data storage~\cite{RoboDB}. It is used to log data of a
robot middleware (Fawkes and ROS) and allows, for example, fault analysis and performance
evaluation. To achieve this, it taps into the messaging infrastructure
in ROS and the blackboard interfaces in Fawkes to store the data one
to one utilizing the flexibility of the schema free MongoDB. The
logged data can later be queried by the robot or a developer. This
allows better evaluation and fault analysis because the data would
otherwise be disposed and logging of the whole
Data-Information-Knowledge-Wisdom (DIKW) hierarchy~\cite{DIKW} enables
retracing processes, such as from point clouds (data) to cluster
positions (information) to object positions (knowledge) combined from
multiple clues to learning results (wisdom). This work already
implemented a basic connection to MongoDB in Fawkes for storing
documents and executing queries, which can be used in this thesis. It
also showed that MongoDB is a good choice because it can handle
querying a large database and writing a lot of data efficiently while
being flexible how and which kind of data is being represented.
However this work lacks some important concepts needed for a robot
memory as intended in this thesis. It is intended as logging facility
and not as working memory holding a world model for multiple planners
and reasoners. Therefore updating and querying mechanisms for planners
and reasoners are missing as well as triggers, a knowledge computable
on demand, multi-robot synchronization, a suitable front-end, ad
integration into knowledge based systems.

\section{Open-EASE}
\label{sec:openease}
Open-EASE is a knowledge processing service for robots and AI
researchers~\cite{OpenEASE}. It is based on and combines
KnowRob~\cite{KnowRob} and the Generic Robot Database with MongoDB in
ROS~\cite{RoboDB}. It is designed as a web service accessible by
robots via a socket connection and by humans via a web browser. It
uses KnowRob for commonsense reasoning and storing the world model of
a robot. The Generic Robot Database is used to log data about robots
or humans performing manipulation tasks. This data can then be
accessed by using the virtual knowledge base concept to learn from
human demonstration and analyze faults. The focus of Open-EASE also is
on providing the recorded data and access to KnowRob to humans using
the web interface. This allows using Open-EASE as eLearning tool for
students, who can explore and experiment with the data and system on
the robot. Furthermore it fosters reproducing experiment results
(e.g. for reviews) and visualizing them. Although the system is
accessible from multiple users and robots, which can query the same
Open-EASE instance, it is focused on single robot data and non
collaborative processes because each user operates in his private
container with individual knowledge base. This makes sense in the
context of student exercises, but is not suitable for multi-robot
systems or multiple knowledge based systems sharing knowledge. If the
system is also usable or being extended for cooperating robots or
planners exchanging knowledge over the web service is not mentioned in
current publications or on the project
website\footnote{\url{http://www.open-ease.org/}}.

\section{MongoDB extensions}
\begin{itemize}
\item trigger implementation
\item multi-master sync (pro/con)
\end{itemize}
There also exist extensions of MongoDB which implement a
\emph{Multi-Master Replication}. This would enable writing to each
instance immediately and synchronizing the changes
later. \todo{source} This allows faster writing especially if the
Master is not available for some time but requires to resolve write
conflicts and handle them in the application. Thus Master-Slave
Replication is more convinient for the applicaiton.  There are also
extensions of MongoDB providing so called \emph{Trigers} which can
notify an application if something in the database changes as
previouly defined. \todo{source} These Triggers use the oplog of
MongoDB to find changes and are useful for planners. For example if a
global planner determined a plan that is being executed and wants to
be notified, if a condition changes that makes the plan invalid, to
replan.

