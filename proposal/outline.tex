\documentclass[a4paper,11pt]{article}

%%%%%%%%%
%%uses%%
%%%%%%%%%
\usepackage[utf8]{inputenc}
%\usepackage[ngerman]{babel}
%\usepackage{a4wide}
\usepackage[margin=3.0cm, top=3.0cm, bottom=3.0cm]{geometry}
\usepackage{setspace}
\usepackage{graphicx}
\usepackage{amssymb} 
\usepackage{amsmath}
\usepackage{mathtools}
\usepackage{footnote}
\usepackage{caption}
\usepackage{color}
\usepackage[hidelinks]{hyperref}
\usepackage{cite}
\usepackage{setspace}
\usepackage{todonotes}


%%%%%%%%%
%%Title%%
%%%%%%%%%

\author{Frederik Zwilling\\304314}
\title{Outline: A robot memory for Fawkes}
\begin{document}
\maketitle
\tableofcontents
\newpage

%%%%%%%%%
%%Text%%
%%%%%%%%


\section{Abstract}

\section{Introduction}
\begin{itemize}
\item Motivation Knowledge Representation on Cyber Physical Systems
\item Problem description
  \begin{itemize}
  \item Worldmodel exchange between resoners/planners
  \item Worldmodel exchange between robots
  \item Long time storage
  \end{itemize}
\item Idea Robot Memory
\item Application RCLL PDDL-CLIPS
\item Application @Home/PR2
\end{itemize}

\section{Background}
\subsection{RoboCup}
\subsubsection{RCLL}
\subsubsection{@Home}
\subsection{Fawkes Robot Software Framework}
\begin{itemize}
\item Component based approach
\item Blackboard
\item Comparison to ROS
\end{itemize}
\subsection{Planners and Reasoners}
\begin{itemize}
\item CLIPS
\item PDDL
\item Golog
\item MoveIt
\end{itemize}
\subsection{Database}
\begin{itemize}
\item MongoDB
\item Query Language
\item Comparison to using CLIPS, KnowRob
\item Comparison to other DBs (SQL, Graph DBs, other Document DBs)
\end{itemize}

\section{Related Work}
\subsection{KnowRob}
\begin{itemize}
\item Common sense Resoning
\item Concepts Ontologies, Triple Store, Virtual Knowledge base
\item Prolog implementation
\item Pro/Contra
\end{itemize}
\subsection{MongoDB Logging}
\subsection{More}
\todo[inline]{More}
\begin{itemize}
\item Blackboard
\item RoboEarth
\item Robot Memory... Robot Databases... Robot Knowledge Storage...
\end{itemize}


\section{Approach}
\subsection{Goals}
\begin{itemize}
\item Same worldmodel for reasoners/planners
\item Distributed for robot team
\item Long time storage (persistent, different kinds of long time knowledge, decay?)
\item Triggers
\item Interrupt storage?
\item Beliefs/Confidences?
\item Versatility (virtual knowledge base)
\item Temporal/Spatial grounding
\item Common sense knowledge
\end{itemize}
\subsection{Architecture}
\begin{itemize}
\item Data representation
\item Component interactions (diagram)
\item Data acquisition
\end{itemize}


\subsection{Implementation}
\subsubsection{Robot Memory}
\begin{itemize}
\item Data Representation
\item MongoDB query language
\item Trigger
\item Multi-robot synchronization
\end{itemize}
\subsubsection{Planner/Reasoner}
\begin{itemize}
\item Example for using the Robot Memory
\item Queries/Register Triggers
\item Build initial domain
\item When to replan
\end{itemize}

\subsection{Evaluation}
\subsubsection{Application}
\begin{itemize}
\item RCLL PDDL-CLIPS
\item RCLL between bots
\item @Home
\end{itemize}
\subsubsection{Efficiency-Scalability}
\subsubsection{Expressiveness}
\subsubsection{Versatility}
\subsubsection{Software development expandability, interfacing}
\subsection{Schedule}
\begin{itemize}
\item Timetable
\end{itemize}

\section{Summery}
\begin{itemize}
\item Challenges
\item Impact
\end{itemize}

\bibliographystyle{plain}
\bibliography{references}

\end{document}
