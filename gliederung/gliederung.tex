\documentclass[a4paper,11pt]{article}

%%%%%%%%%
%%uses%%
%%%%%%%%%
\usepackage[utf8]{inputenc}
%\usepackage[ngerman]{babel}
%\usepackage{a4wide}
\usepackage[margin=3.0cm, top=3.0cm, bottom=3.0cm]{geometry}
\usepackage{setspace}
\usepackage{graphicx}
\usepackage{amssymb} 
\usepackage{amsmath}
\usepackage{mathtools}
\usepackage{footnote}
\usepackage{caption}
\usepackage{color}
\usepackage[hidelinks]{hyperref}
\usepackage{cite}
\usepackage{setspace}



%%%%%%%%%
%%Title%%
%%%%%%%%%

\author{Frederik Zwilling 304314}
\title{Outline: A Document-Oriented Robot Memory for Knowledge Sharing and Hybrid Reasoning on Mobile Robots}
\begin{document}
\maketitle
\tableofcontents
\newpage

%%%%%%%%%
%%Text%%
%%%%%%%%

%\abstract{This is the abstract.}


\section{Introduction}
\begin{itemize}
\item Motivation Knowledge Representation on Cyber Physical Systems
\item Problem description
  \begin{itemize}
  \item Worldmodel exchange between resoners/planners
  \item Worldmodel exchange between robots
  \item Long time storage
  \end{itemize}
\item Idea Robot Memory
\item Application RCLL PDDL-CLIPS
\item Application @Home/PR2
\item Outline
\end{itemize}


\section{Background}

\subsection{Context}
\subsubsection{Mobile Robotics}
\begin{itemize}
\item See tnthesis
\end{itemize}
\subsubsection{Memory Storage}
\begin{itemize}
\item dik-hierarchy
\item stored in RAM (in reasoner language)
\item stored on harddrive (databases)
\item cloud (RoboEarth) (move into RW?)
\end{itemize}
\subsubsection{Reasoning}
\begin{itemize}
\item General
\item Hybrid reasoning
\end{itemize}

\subsection{RoboCup}
\begin{itemize}
\item In General
\end{itemize}
\subsubsection{RCLL}
\subsubsection{@Home}

\subsection{Robot Software Frameworks}
\subsubsection{ROS}
\subsubsection{Fawkes}

\subsection{Databases}
\subsubsection{SQL and Document-Oriented}
\subsubsection{Document-Oriented Implementations}
\begin{itemize}
\item MongoDB
\item ArangoDB
\item CouchDB
\end{itemize}

\subsection{Planners and Reasoners}
\begin{itemize}
\item CLIPS Rules Engine
\item Planning Domain Definition Language
\item Motion Planners
\end{itemize}


\section{Related Work}

\subsection{KnowRob}
\subsection{OpenRobots Ontology}
\subsection{Generic Robot Database with MongoDB}
\subsection{Open-EASE}


\section{Approach}
\begin{itemize}
\item Principles/Goals
\end{itemize}
\subsection{Representation/Mathematical Formulation}
\begin{itemize}
\item 
\end{itemize}
\subsection{Planner Integration}
\begin{itemize}
\item General task
\item reasoner, planner, motion planner
\end{itemize}
\subsubsection{Computables}
\begin{itemize}
\item 
\end{itemize}
\subsubsection{Trigger}
\begin{itemize}
\item 
\end{itemize}
\subsubsection{MultiRobot Synchronization}
\begin{itemize}
\item 
\end{itemize}
\subsection{Architecture}
\begin{itemize}
\item 
\end{itemize}


\section{Implementation}
\subsection{Database Integration}
\subsection{RobotMemory Features}
\subsection{Planner/Reasoner Integration}
\subsection{RCLL}
\subsection{@Home/BlocksWorld}


\section{Evaluation}
\subsection{Blocks-World}
\subsection{RCLL Sync}
\subsection{Global Planner PDDL/ASP}
\subsection{Quantitative}


\section{Summery and Future Work}
\subsection{Future Work}
\subsection{Summery}

\bibliographystyle{plain}
\bibliography{references}

\end{document}
